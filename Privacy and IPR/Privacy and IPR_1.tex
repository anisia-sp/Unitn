\documentclass{article}

\usepackage{listings}

\title{Privacy and IPR}
\date{2019-10-02}

\begin{document}
   \maketitle 
   \newpage
   \pagenumbering{roman}

\section{Lecture 2}

\subsection{Recap}
\begin{itemize}
\item Lecture 1:
\item An introducrtion to the meaning of law througth history
\end{itemize}

\subsection{Rationality}

\begin{itemize}
\item Judges and lawyers need to apply the already written rules and laws but use their knowledge and experience as well
\item In the past God represented the basis of rationality
\item So since God is rational and he establishes the rules, the rules are rational as well
\item After the French revolution, rationality of rules started being based on human rationality, the so-called "Principal of Rationality"
\item Rational: every possible legal problem has a unique solution based on the code in power 
\end{itemize}

\subsection{Legal system}
\begin{itemize}
\item The legal system is considered to be perfect 
\item The legal body must understand the sytem 
\item For any new rules, the legal body's job is to analse them in such way that they fit perfectly in the system
\item If they don't, that means that there is a problem in understanding the system 
\end{itemize}

\subsection{}
\begin{itemize}
\item Natural law: a body of unchanging moral principles regarded as a basis for all human conduct 
\item Natural law is above any individual human intervention 
\item Positive law: statutes which have been laid down by a legislature, court, or other human institution and can take whatever form the authors want
\item Kelsen: foundation rule
\item Example of the above: the Italian constitution 
\item Building blocks of the Western legal tradition: property 
\item Another one is contract
\end{itemize}

\subsection{Property}
\begin{itemize}
\item Medieval tradition: land -> agriculture to produce food, to feed animals or grow trees to use the wood 
\item These uses are different, but then they weren't seen as such by the legal system
\item Tenure: the conditions under which land or buildings are held or occupied 
\item There was no individual ownership in the medieval times 
\item Renaissance: new social class, individual and collective enterprises that tread internationally as well
\item The new class wanted access to property too
\item Property changes ownership from linage to wealth - especially after the French revolutiion
\item Hegel: property is something naturally embodied in the human spirit 
\item Ownership is the expression of the link between an individual and the external actions in his life 
\item Property is becoming the core of the new order 
\item The success and the social status of an individual is measured by the property they own
\item The protection of property becomes the focus of the legal system
\item Property is connected to freedom if seen as a 'space' where the individual can do wahtever they want 
\item Basis of property is natural law 
\item Modern property: it is regulated sector by sector 
\item Liberalism: a political and moral philosophy based on liberty, consent of the governed, and equality before law - Western legal tradition
\item Real property: in rem right - right associated with a property, not based on any personal relationship 
\item In personam right: a personal right attached to a specific person, such as contract rigths, a tort award against a defendant, or a license 
\item In rem rights: property rigths enforceable against the entire world whereas an in personam judgment binds only the litigants 
\item Traditionally, property was a right over physical things 
\item Intellectual property: contradiction to the definition of property, right to exclude from its use 
\item Property: best example of exclusive right, highest level of protection, right to block any interfering action or in a lower level get compensation for the damage caused by interference (liability)
\end{itemize}

\end{document}
\documentclass{article}


\title{Privacy and IPR}
\date{2019-10-9}

\begin{document}
   \maketitle
   \newpage
   \pagenumbering{roman}

\section{Lecture 3}

\subsection{Intro}
\begin{itemize}
\item Growth of intellectual property after the 60s
\item Legal institutions are involved with intellectual property 
\item Copyright, Patters, Trademarks
\item Copyright and Patters are very diffferent 
\item Trademark: protection of consumers and the producers' rigths 
\item Inclusive right is what connects these institutions but apart from that they have many differences 
\item Trademark - Domain names 
\item Although in trademark more than one company can use the same name as long as they don't share the same market in domains that can't happen 
\end{itemize}

\subsection{}
\begin{itemize}
\item Intellectual property involved with products can also be called industrial property 
\item e.x. software patters 
\end{itemize}

\subsection{Copyright} 
\begin{itemize}
\item We think of it as the authors right 
\item The statute of Anne (1710): the first statute to provide for copyright regulated by the government and courts, rather than by private parties
\item U.S. Constitution - Copyright Act (1790): modeled off Britain's Statute of Anne, the new law is relatively limited in scope, protecting books, maps, and charts for only 14 years with a renewal period of another 14 years
\item There is a contradiction between the exclusive right of the owner and the publc interest, so we set limitation to that right 
\end{itemize}

\subsection{Origins of Copyright} 
\begin{itemize}
\item The stationer's copyright: The Stationer's Register was a record book that allowed publishers to document their right to produce a particular printed work, and constituted an early form of copyright law. 
\item The Charter of the stationers' company: was formed in 1403; it received a royal charter in 1557. It held a monopoly over the publishing industry and was officially responsible for setting and enforcing regulations until the enactment of the Statute of Anne, also known as the Copyright Act of 1710
\item The private association had the power to manage and regulate the profession and had the right to add newcomers 
\item The association was recognised by the kings and queens by giving them a charter 
\item According to Patterson: the stationers created their copyright, shaped it to their ends, and its kept control for themselves 
\item Registartion of the book's titles (mandatory since 1662 with the licensing act), perpetual monopoly, policies to avoid competition
\item Authors: someone who create original ideas 
\end{itemize}

\subsection{End of Stationer's Copyright}
\begin{itemize}
\item Enlightment
\item Growing consensus against monopolies and the problem of public domain: Locke's critique brought down the stationers' copyright 
\item Comedy : use of comments to explain the context 
\item The commentors' work was not recognised or appreciated 
\end{itemize}

\subsection{The Statute of Anne}
\begin{itemize}
\item The publishers had adopted a negative image so the portection of the authors' rights was required 
\item It was basically a compromise so that the stationers' copyright didn't lose all thier power 
\item Creation of a public domain of literary works 
\end{itemize}

\subsection{Public Domain }
\begin{itemize}
\item The state of belonging or being available to the public as a whole, especially through not being subject to copyright or other legal restrictions
\item Public Resource 
\item No one has the inclusive right on it 
\item Everyone had the right to take form it 
\item When the exclusive rigths had expired , the intellectual work belonged to the public 
\item The stationers didn't like that 
\item The book is only protected for a limited time 
\end{itemize}
\end{document} 


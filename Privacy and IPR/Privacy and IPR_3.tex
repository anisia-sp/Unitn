\documentclass{article}

\usepackage{amsmath}

\title{Privacy and IPR}
\date{2019-10-10}

\begin{document}
   \maketitle
   \newpage
   \pagenumbering{roman}

\section{Lecture 4}

\subsection{Recap}
\begin{itemize}
\item Modern copright: exlcusive right given to authors is not a reaction to recent technological achievements
\item Congress should have the right to give authors an exclusive right for a limited time 
\item Foundation of the exclusive right: progress of  learning, writing -> public interest  (main difference with property)
\item Modern copright: based on the authors' right and not the publishers' rigths created the public domain
\item stationers wanted unlimited exclusive rights to publishers, despite the fact that authors would sell them the right for a limited time 
\item 1787 , 1790 U.S. Copyright Act -> copy of the statute of Anne
\end{itemize}

\subsection{Extention and globalization of copyright law}
\begin{itemize}
\item From literary work to other intellectual property 
\item Copyright: way of regulating intellectual production 
\item Globalization: copyright ->private law institution legalised worldwide 
\item The process of reciprocity: copyright granted to a foreign author only through reciprocity 
\item Next step: creation of an international association 
\item 1886: First international multilateral treaty 
\item U.S. was using the EU regulations  - the were importing culture 
\item The first legislation was not protecting foreign authors in the U.S.
\item It was possible to make copies without paying anything to foreign authors, unless they lived in the U.S. and were registared with a U.S. publisher
\item 1994: Annex 1C to the Marakesh Agreement 
\item 1996: WIPO Copright Treaty 
\item 1998: U.S. became part of the Berne Convention (meaning that the creation of the intellectual work gives the protection , given that specific requirements were covered)
\item They became exporters of culture so their views on intellectual property changed 
\end{itemize}

\subsection{Compatibility between the scope and the means}
\begin{itemize}
\item Now copyright lasts for 70 tears after the authors' death 
\item Importance of the limit to the right to exclude
\item Limits: extension and duration
\end{itemize}

\subsection{Berne Convention ("Protected works")}
\begin{itemize}
\item Berne Convention: is an international agreement governing copyright, which was first accepted in Berne, Switzerland, in 1886
\item Gave a list of what the proetcted works were
\item Nothing about originality 
\end{itemize}

\subsection{Title 17 U.S. Code} 
\begin{itemize}
\item Nature of the intellectual work
\item Open to futurre creation 
\item The creator can be a machine 
\item A copy is a copy only if a human can see it without the use of a machine
\item Computer programs: the source code, written in high level language can be protected by the copyright law but not the object code 
\item There is also the case of originality  (also in the Italian copyright law)
\end{itemize}

\subsection{U.S. Constitution} 
\begin{itemize}
\item The protection is given to authors 
\item The output of a computer program is not produced by an author but by a machine, so not protected 
\item We should also see the case of protection of compilation of non-original facts
\item Protection of databases: the ordering of the collected data is original creation, work of authorship
\item Sui generis right : protects the creation of databses only if the creation requires money and lasts only for 50 years after its creation 
\end{itemize}

\subsection{Expression/ Idea Dichotomy}
\begin{itemize}
\item Form of expression vs idea 
\item Copyright protects the expression not the idea 
\end{itemize}


\subsection{Limitations on exclusive rights: Fair use}
\begin{itemize}
\item Purpuse of use - profit or not 
\item Nature of copyrighted work
\item Amount and substantiality of the work
\item Effect of the use upon the potential market 
\end{itemize}

\subsection{Italy (Copyright Law)}
\begin{itemize}
\item Specific exceptions and not a general clause 
\end{itemize}


\end{document}
\documentclass{article}

\usepackage{amsmath}

\title{Privacy and IPR}
\date{2019-10-10}

\begin{document}
   \maketitle
   \newpage
   \pagenumbering{roman}

\section{Lecture 4}

\subsection{Recap}
\begin{itemize}
\item Modern copyright: exlcusive right given to authors is not a reaction to recent technological achievements
\item Foundation of the exclusive right: progress of  learning, writing -> public interest  (main difference with property)
\item The modern copyright based on the authors' right, not the publishers' rights, created the public domain
\item Stationers wanted unlimited exclusive rights to publishers, despite the fact that authors would sell them the right for a limited time 
\end{itemize}

\subsection{Extention and globalization of copyright law}
\begin{itemize}
\item From literary work to other intellectual property 
\item Copyright: way of regulating intellectual production 
\item Globalization: copyright ->private law institutions legalised worldwide 
\item The process of reciprocity: copyright granted to a foreign author only through reciprocity 
\item Next step: creation of an international association 
\item 1886: First international multilateral treaty 
\item U.S. was using the EU regulations  - they were importing culture 
\item The first legislation was not protecting foreign authors in the U.S.
\item It was possible to make copies without paying anything to foreign authors, unless they lived in the U.S. and were registared with a U.S. publisher
\item The TRIPS Agreement is Annex 1C of the Marrakesh Agreement establishing the World Trade Organization, signed in Marrakesh, Morocco on 15 April 1994. This agreement puts in place a multilateral framework for addressing intellectual property issues in international commercial transactions.
\item The WIPO Copyright Treaty (WCT) is a special agreement under the Berne Convention which deals with the protection of works and the rights of their authors in the digital environment. In addition to the rights recognized by the Berne Convention, they are granted certain economic rights.  The Treaty also deals with two subject matters to be protected by copyright: (i) computer programs, whatever the mode or form of their expression; and (ii) compilations of data or other material ("databases")
\item 1998: U.S. became part of the Berne Convention
\item U.S. becomes exporter of culture so the views on intellectual property changed 
\end{itemize}

\subsection{Compatibility between the scope and the means}
\begin{itemize}
\item Now copyright lasts for 70 tears after the authors' death 
\item Importance of the limit to the right to exclude
\item Limits: extension and duration
\end{itemize}

\subsection{Berne Convention ("Protected works")}
\begin{itemize}
\item Berne Convention: is an international agreement governing copyright, which was first accepted in Berne, Switzerland, in 1886
\item it introduced the concept that a copyright exists the moment a work is "fixed", rather than requiring registration
\item enforces a requirement that countries recognize copyrights held by the citizens of all other parties to the convention
\item includes a number of specific copyright exceptions, scattered in several provisions due to the historical reason of Berne negotiations
\item Nothing about originality 
\end{itemize}

\subsection{Title 17 U.S. Code} 
\begin{itemize}
\item the United States Code that outlines United States copyright law. It was codified into positive law on July 30, 1947
\item United States copyright law is intended to encourage the creation of art and culture by rewarding authors and artists with a set of exclusive rights. Copyright law grants authors and artists the exclusive right to make and sell copies of their works, the right to create derivative works, and the right to perform or display their works publicly. These exclusive rights are subject to a time limit, and generally expire 70 years after the author's death
\item Nature of the intellectual work
\item Open to future creation 
\item A copy is a copy only if a human can see it without the use of a machine
\item Computer programs: the source code, written in high level language can be protected by the copyright law but not the object code 
\item There is also the case of originality  (also in the Italian copyright law)
\end{itemize}

\subsection{U.S. Constitution} 
\begin{itemize}
\item The protection is given to authors 
\item The output of a computer program is not produced by an author but by a machine, so not protected 
\item We should also see the case of protection of compilation of non-original facts
\item Protection of databases: the ordering of the collected data is original creation, work of authorship
\item Sui generis right : an exclusive right that protects databases against unauthorised extraction and re-utilisation of their content. It is distinct and independent from copyright, which protects original works (protects the creation of databses only if the creation requires money and lasts only for 50 years after its creation)
\end{itemize}

\subsection{Expression/ Idea Dichotomy}
\begin{itemize}
\item was formulated to ensure that the manifestation of an idea (i.e. an expression) is protected rather than the idea itself. An idea is the formulation of thought on a particular subject whereas an expression constitutes the implementation of the said idea
\item Form of expression vs idea 
\item Copyright protects the expression not the idea 
\end{itemize}


\subsection{Limitations on exclusive rights: Fair use}
\begin{itemize}
\item The purpose and character of the use
\item The nature of the work
\item The amount and substantiality of the portion used in relation to the work as a whole
\item The effect of the use on the market or potential market for the original work
\end{itemize}

\subsection{Italy (Copyright Law)}
\begin{itemize}
\item Specific exceptions and not a general clause 
\end{itemize}


\end{document}